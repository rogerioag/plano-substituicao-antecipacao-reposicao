\documentclass[a4paper, 10pt]{report}
\usepackage{utfpr}

\usepackage[T1]{fontenc}

\usepackage[utf8]{inputenc}

%\usepackage{lipsum}
\usepackage{scalefnt}

\begin{document}
\clearpage

\deptocoord{DACOM}{COCIC}{Curso de Bacharelado em Ciência da Computação}

\cabecalho{SUBSTITUIÇÃO / REPOSIÇÃO / ANTECIPAÇÃO DE AULAS}

% Escolha o tipo: Substituição, Reposição ou Antecipação.
% \type{Substituição} % Não aparece a listagem de alunos.
% \type{Reposição}
\type{Antecipação}

% Especifique um Motivo. Divida o texto em quatro linhas ou deixe em branco.
\motivolinhaUm{Viagem a Congresso Nacional}
\motivolinhaDois{Viagem a Congresso Nacional}
\motivolinhaTres{Viagem a Congresso Nacional}
\motivolinhaQuatro{Viagem a Congresso Nacional}

% Indique o nome do professor solicitante.
\solicitante{Testerson dos Santos}{COCIC}{\today}

% Cabeçalho da mudança. Não precisa alterar.
\cabecalhomudanca

% Especifique os dados das mudanças de aulas.
% Data da Aula | Data Proposta | Horário | Código Disciplina-Turma | Sala | Nome do Prof. Substituto.
\mudanca{25/12/2017}{01/01/2018}{N1-N3}{PD36O-IC6A}{E-102}{Jamal Hem Fhazer}
\mudanca{03/03/2018}{10/04/2018}{T1-T2}{IXI03-IC3E}{E-003}{Olairson Tokada Ioshi}
\mudanca{25/12/2017}{01/01/2018}{N1-N3}{PD36O-IC6A}{E-102}{Testelvina da Silva}
\mudanca{25/12/2017}{01/01/2018}{N1-N3}{PD36O-IC6A}{E-102}{Krzysztof de Souza}

% Imprime o quadro de parecer.
\parecer

% De acordo dos substitutos.
\deacordo

% Quadro de observações.
\observacoes

% Lista dos Alunos. Se precisar mais de uma folha, só chamar o comando novamente.
% A lista é impressa somente para Reposição e Antecipação. 
\listaalunos

\listaalunos

\listaalunos

\end{document}